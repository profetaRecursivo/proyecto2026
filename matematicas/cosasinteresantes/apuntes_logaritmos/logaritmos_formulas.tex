\documentclass[11pt]{article}

% Paquetes necesarios
\usepackage{amsthm}
\usepackage{amsmath}
\usepackage{amssymb}
\usepackage{geometry}
\usepackage{tikz}
\usepackage{pgfplots}
\usepackage{xcolor}
\usepackage{multicol}
\usepackage{enumitem}
\usepackage{mdframed}

\pgfplotsset{compat=1.18}

% Configuración de márgenes
\geometry{a4paper, margin=0.8in}

% Colores personalizados
\definecolor{boxcolor}{RGB}{240,248,255}
\definecolor{titlecolor}{RGB}{0,51,102}

% Entorno para fórmulas importantes
\newmdenv[
    backgroundcolor=boxcolor,
    linecolor=titlecolor,
    linewidth=2pt,
    roundcorner=5pt,
    skipabove=10pt,
    skipbelow=10pt
]{importantbox}

\title{\textbf{\Huge Cheatsheet de Logaritmos} \\ 
       \large Propiedades, Identidades y Aplicaciones Matemáticas}
\author{Mathematical Reference Guide}
\date{}

\begin{document}

\maketitle

\begin{abstract}
Este documento contiene una colección exhaustiva de propiedades, identidades y aplicaciones de logaritmos. Incluye cambios de base, ecuaciones logarítmicas y exponenciales, series, límites, y aplicaciones en diversas áreas de las matemáticas.
\end{abstract}

\tableofcontents

\newpage

\section{Definiciones Fundamentales}

\subsection{Definición de Logaritmo}

\begin{importantbox}
\[
\boxed{y = \log_b(x) \iff b^y = x}
\]
\end{importantbox}

Donde:
\begin{itemize}
    \item $b$ es la \textbf{base} ($b > 0$, $b \neq 1$)
    \item $x$ es el \textbf{argumento} ($x > 0$)
    \item $y$ es el \textbf{logaritmo} o \textbf{exponente}
\end{itemize}

\subsection{Logaritmos Especiales}

\textbf{Logaritmo natural (base $e$):}
\[
\ln(x) = \log_e(x)
\]

\textbf{Logaritmo decimal (base 10):}
\[
\log(x) = \log_{10}(x)
\]

\textbf{Logaritmo binario (base 2):}
\[
\lg(x) = \log_2(x)
\]

\subsection{Valores Especiales}

\begin{align*}
\log_b(1) &= 0 \\
\log_b(b) &= 1 \\
\log_b(b^n) &= n \\
b^{\log_b(x)} &= x \\
\log_b\left(\frac{1}{x}\right) &= -\log_b(x)
\end{align*}

\section{Propiedades Fundamentales}

\subsection{Leyes de los Logaritmos}

\begin{importantbox}
\textbf{Producto:}
\[
\boxed{\log_b(xy) = \log_b(x) + \log_b(y)}
\]

\textbf{Cociente:}
\[
\boxed{\log_b\left(\frac{x}{y}\right) = \log_b(x) - \log_b(y)}
\]

\textbf{Potencia:}
\[
\boxed{\log_b(x^n) = n \cdot \log_b(x)}
\]
\end{importantbox}

\subsection{Cambio de Base}

\begin{importantbox}
\[
\boxed{\log_b(x) = \frac{\log_c(x)}{\log_c(b)} = \frac{\ln(x)}{\ln(b)}}
\]
\end{importantbox}

\textbf{Casos especiales:}
\begin{align*}
\log_b(x) &= \frac{1}{\log_x(b)} \\
\log_{a^n}(x) &= \frac{1}{n}\log_a(x) \\
\log_a(b) \cdot \log_b(c) &= \log_a(c) \\
\log_a(b) \cdot \log_b(a) &= 1
\end{align*}

\subsection{Raíces}

\begin{align*}
\log_b(\sqrt[n]{x}) &= \frac{1}{n}\log_b(x) \\
\log_b(\sqrt{x}) &= \frac{1}{2}\log_b(x) \\
\sqrt[n]{b^x} &= b^{x/n}
\end{align*}

\section{Identidades Avanzadas}

\subsection{Identidades con Múltiples Logaritmos}

\[
\log_b(x) + \log_b(y) + \log_b(z) = \log_b(xyz)
\]

\[
\log_b(x) - \log_b(y) - \log_b(z) = \log_b\left(\frac{x}{yz}\right)
\]

\[
n\log_b(x) + m\log_b(y) = \log_b(x^n y^m)
\]

\subsection{Identidades con Cambios de Base}

\[
\log_a(b) \cdot \log_b(c) \cdot \log_c(a) = 1
\]

\[
\log_{a^m}(b^n) = \frac{n}{m}\log_a(b)
\]

\[
\log_{\sqrt{a}}(b) = 2\log_a(b)
\]

\subsection{Relaciones Recíprocas}

\[
\frac{1}{\log_b(a)} = \log_a(b)
\]

\[
\log_a(b) + \log_b(a) = \frac{\log_a^2(b) + 1}{\log_a(b)}
\]

\[
\log_a(b) - \log_b(a) = \frac{\log_a^2(b) - 1}{\log_a(b)}
\]

\section{Logaritmo Natural}

\subsection{Definición como Integral}

\[
\boxed{\ln(x) = \int_1^x \frac{1}{t}\,dt}
\]

\subsection{Propiedades del Logaritmo Natural}

\begin{align*}
\ln(e) &= 1 \\
\ln(1) &= 0 \\
e^{\ln(x)} &= x \\
\ln(e^x) &= x \\
\ln(ab) &= \ln(a) + \ln(b) \\
\ln\left(\frac{a}{b}\right) &= \ln(a) - \ln(b) \\
\ln(a^n) &= n\ln(a)
\end{align*}

\subsection{Derivada e Integral}

\begin{importantbox}
\textbf{Derivada:}
\[
\boxed{\frac{d}{dx}[\ln(x)] = \frac{1}{x}}
\]

\textbf{Integral:}
\[
\boxed{\int \frac{1}{x}\,dx = \ln|x| + C}
\]
\end{importantbox}

\textbf{Derivada de logaritmo en base $b$:}
\[
\frac{d}{dx}[\log_b(x)] = \frac{1}{x \ln(b)}
\]

\textbf{Derivada de funciones logarítmicas:}
\begin{align*}
\frac{d}{dx}[\ln(f(x))] &= \frac{f'(x)}{f(x)} \\
\frac{d}{dx}[\ln|x|] &= \frac{1}{x} \\
\frac{d}{dx}[x\ln(x)] &= \ln(x) + 1
\end{align*}

\section{Series y Expansiones}

\subsection{Serie de Taylor del Logaritmo Natural}

\begin{importantbox}
\[
\boxed{\ln(1+x) = x - \frac{x^2}{2} + \frac{x^3}{3} - \frac{x^4}{4} + \cdots = \sum_{n=1}^{\infty} \frac{(-1)^{n+1}x^n}{n}}
\]
\end{importantbox}

Válida para $|x| \leq 1$ (y $x = 1$).

\textbf{Casos especiales:}
\[
\ln(2) = 1 - \frac{1}{2} + \frac{1}{3} - \frac{1}{4} + \cdots = \sum_{n=1}^{\infty} \frac{(-1)^{n+1}}{n}
\]

\subsection{Otras Expansiones}

\[
\ln\left(\frac{1+x}{1-x}\right) = 2\left(x + \frac{x^3}{3} + \frac{x^5}{5} + \cdots\right) = 2\sum_{n=0}^{\infty} \frac{x^{2n+1}}{2n+1}
\]

\[
\ln(x) = (x-1) - \frac{(x-1)^2}{2} + \frac{(x-1)^3}{3} - \cdots \quad (0 < x \leq 2)
\]

\[
\ln(x) = 2\left[\frac{x-1}{x+1} + \frac{1}{3}\left(\frac{x-1}{x+1}\right)^3 + \frac{1}{5}\left(\frac{x-1}{x+1}\right)^5 + \cdots\right]
\]

\subsection{Productos Infinitos}

\[
\ln(2) = \frac{1}{1 \cdot 2} + \frac{1}{3 \cdot 4} + \frac{1}{5 \cdot 6} + \cdots
\]

\[
e = \lim_{n \to \infty} \left(1 + \frac{1}{n}\right)^n
\]

\section{Límites Importantes}

\subsection{Límites Fundamentales con Logaritmos}

\begin{importantbox}
\[
\boxed{\lim_{x \to 0} \frac{\ln(1+x)}{x} = 1}
\]

\[
\boxed{\lim_{x \to \infty} \frac{\ln(x)}{x} = 0}
\]

\[
\boxed{\lim_{x \to \infty} \frac{\ln(x)}{x^n} = 0 \quad (n > 0)}
\]
\end{importantbox}

\subsection{Límites con Exponenciales}

\[
\lim_{x \to 0} \frac{e^x - 1}{x} = 1
\]

\[
\lim_{x \to 0} \frac{a^x - 1}{x} = \ln(a)
\]

\[
\lim_{n \to \infty} \left(1 + \frac{x}{n}\right)^n = e^x
\]

\[
\lim_{x \to 0} (1+x)^{1/x} = e
\]

\subsection{Límites Indeterminados}

Para evaluar límites de la forma $\lim f(x)^{g(x)}$:

\[
\lim f(x)^{g(x)} = e^{\lim g(x)\ln(f(x))}
\]

\textbf{Formas indeterminadas:}
\begin{itemize}
    \item $1^\infty$: usar $\ln$ y l'Hôpital
    \item $0^0$: usar $\ln$ y l'Hôpital
    \item $\infty^0$: usar $\ln$ y l'Hôpital
\end{itemize}

\section{Ecuaciones Logarítmicas}

\subsection{Ecuaciones Básicas}

\textbf{Forma 1:} $\log_b(x) = c$
\[
\boxed{x = b^c}
\]

\textbf{Forma 2:} $\log_b(f(x)) = \log_b(g(x))$
\[
\boxed{f(x) = g(x)}
\]

\textbf{Forma 3:} $\log_b(x) + \log_b(y) = c$
\[
xy = b^c
\]

\subsection{Ecuaciones Logarítmicas Compuestas}

\textbf{Tipo:} $a\log_b(x) + c = d$
\[
x = b^{(d-c)/a}
\]

\textbf{Tipo:} $\log_b(x) + \log_b(x+a) = c$
\[
x(x+a) = b^c
\]

\textbf{Tipo:} $\log_b(x) - \log_b(x-a) = c$
\[
\frac{x}{x-a} = b^c
\]

\subsection{Cambio de Variable}

Para ecuaciones como $(\log_b x)^2 + a\log_b x + c = 0$:

Hacer $y = \log_b x$, resolver la ecuación cuadrática, luego $x = b^y$.

\section{Ecuaciones Exponenciales}

\subsection{Ecuaciones Básicas}

\textbf{Forma 1:} $b^x = c$
\[
\boxed{x = \log_b(c) = \frac{\ln(c)}{\ln(b)}}
\]

\textbf{Forma 2:} $a \cdot b^x = c$
\[
x = \log_b\left(\frac{c}{a}\right)
\]

\textbf{Forma 3:} $b^x = b^y$
\[
x = y
\]

\subsection{Ecuaciones con Bases Diferentes}

\textbf{Tipo:} $a^x = b^x$
\[
x = 0 \quad \text{o} \quad a = b
\]

\textbf{Tipo:} $a^x = b^y$
\[
x\ln(a) = y\ln(b)
\]

\textbf{Tipo:} $a^{f(x)} = b^{g(x)}$
\[
f(x)\ln(a) = g(x)\ln(b)
\]

\subsection{Ecuaciones Exponenciales Cuadráticas}

Para $a^{2x} + ba^x + c = 0$:

Hacer $y = a^x$, resolver $y^2 + by + c = 0$, luego $x = \log_a(y)$.

\section{Desigualdades Logarítmicas}

\subsection{Desigualdades Básicas}

Si $b > 1$:
\[
\log_b(x) < \log_b(y) \iff x < y
\]

Si $0 < b < 1$:
\[
\log_b(x) < \log_b(y) \iff x > y
\]

\subsection{Desigualdades Importantes}

\begin{importantbox}
\[
\boxed{\ln(x) \leq x - 1 \quad \text{para todo } x > 0}
\]
\end{importantbox}

Igualdad solo cuando $x = 1$.

\[
\frac{x-1}{x} \leq \ln(x) \leq x - 1 \quad (x \geq 1)
\]

\[
\ln(x) < \sqrt{x} \quad (x > 1)
\]

\[
x\ln(x) - x < \ln(x!) < x\ln(x) \quad (x > 1)
\]

\subsection{Desigualdad de Bernoulli Logarítmica}

\[
\ln(1 + x) \geq \frac{x}{1 + x} \quad (x > -1)
\]

\[
\ln(1 + x) \leq x \quad (x > -1)
\]

\section{Aproximaciones}

\subsection{Aproximaciones para $x$ Pequeño}

\[
\ln(1+x) \approx x - \frac{x^2}{2} \quad (|x| \ll 1)
\]

\[
\ln(1+x) \approx x \quad (|x| \ll 1)
\]

\[
e^x \approx 1 + x \quad (|x| \ll 1)
\]

\[
(1+x)^n \approx 1 + nx \quad (|x| \ll 1)
\]

\subsection{Aproximaciones para Grandes Valores}

\[
\ln(n!) \approx n\ln(n) - n \quad \text{(Aproximación de Stirling)}
\]

\[
\ln(n!) \approx n\ln(n) - n + \frac{1}{2}\ln(2\pi n) \quad \text{(Stirling mejorada)}
\]

\[
\log_2(n!) \approx n\log_2(n) - n\log_2(e)
\]

\section{Funciones Relacionadas}

\subsection{Función Exponencial}

\begin{importantbox}
\[
\boxed{e^x = \sum_{n=0}^{\infty} \frac{x^n}{n!} = 1 + x + \frac{x^2}{2!} + \frac{x^3}{3!} + \cdots}
\]
\end{importantbox}

\textbf{Propiedades:}
\begin{align*}
e^{x+y} &= e^x \cdot e^y \\
e^{x-y} &= \frac{e^x}{e^y} \\
(e^x)^n &= e^{nx} \\
e^0 &= 1 \\
e^1 &= e \\
e^{-x} &= \frac{1}{e^x}
\end{align*}

\subsection{Función Potencia vs Exponencial vs Logaritmo}

\textbf{Jerarquía de crecimiento (para $x \to \infty$):}
\[
\log(x) \ll x^a \ll b^x \ll x! \ll x^x
\]

Para cualquier $a > 0$ y $b > 1$.

\textbf{Límites comparativos:}
\begin{align*}
\lim_{x \to \infty} \frac{\ln(x)}{x^a} &= 0 \quad (a > 0) \\
\lim_{x \to \infty} \frac{x^a}{b^x} &= 0 \quad (b > 1) \\
\lim_{x \to \infty} \frac{b^x}{x!} &= 0
\end{align*}

\section{Logaritmos en Diferentes Contextos}

\subsection{Logaritmo Complejo}

Para $z = re^{i\theta}$ (forma polar):
\[
\ln(z) = \ln(r) + i\theta = \ln|z| + i\arg(z)
\]

\textbf{Fórmula de Euler:}
\[
e^{i\theta} = \cos(\theta) + i\sin(\theta)
\]

\[
\ln(-1) = i\pi
\]

\[
\ln(i) = i\frac{\pi}{2}
\]

\subsection{Logaritmo Discreto}

En teoría de números, el logaritmo discreto de $a$ en base $b$ módulo $n$ es el menor entero $x$ tal que:
\[
b^x \equiv a \pmod{n}
\]

No existe una fórmula general cerrada (problema computacionalmente difícil).

\subsection{Logaritmo Iterado}

\[
\log^*(n) = \begin{cases}
0 & \text{si } n \leq 1 \\
1 + \log^*(\log n) & \text{si } n > 1
\end{cases}
\]

Es el número de veces que se debe aplicar $\log$ para obtener un valor $\leq 1$.

\textbf{Ejemplos:}
\begin{align*}
\log^*(2) &= 1 \\
\log^*(16) &= 3 \\
\log^*(65536) &= 4 \\
\log^*(2^{65536}) &= 5
\end{align*}

Crece extremadamente lento.

\section{Identidades Trigonométricas con Logaritmos}

\subsection{Relaciones con Funciones Hiperbólicas}

\begin{align*}
\sinh(x) &= \frac{e^x - e^{-x}}{2} \\
\cosh(x) &= \frac{e^x + e^{-x}}{2} \\
\tanh(x) &= \frac{e^x - e^{-x}}{e^x + e^{-x}} = \frac{e^{2x} - 1}{e^{2x} + 1}
\end{align*}

\textbf{Inversas:}
\begin{align*}
\sinh^{-1}(x) &= \ln(x + \sqrt{x^2 + 1}) \\
\cosh^{-1}(x) &= \ln(x + \sqrt{x^2 - 1}) \quad (x \geq 1) \\
\tanh^{-1}(x) &= \frac{1}{2}\ln\left(\frac{1+x}{1-x}\right) \quad (|x| < 1)
\end{align*}

\subsection{Identidades de Euler}

\begin{align*}
\sin(x) &= \frac{e^{ix} - e^{-ix}}{2i} \\
\cos(x) &= \frac{e^{ix} + e^{-ix}}{2}
\end{align*}

\section{Aplicaciones Especiales}

\subsection{Entropía de Shannon}

\[
H(X) = -\sum_{i} p_i \log_2(p_i)
\]

Mide la cantidad promedio de información.

\subsection{Magnitud en Escalas Logarítmicas}

\textbf{Decibeles (dB):}
\[
L = 10\log_{10}\left(\frac{I}{I_0}\right)
\]

\textbf{Escala de Richter:}
\[
M = \log_{10}\left(\frac{A}{A_0}\right)
\]

\textbf{pH:}
\[
\text{pH} = -\log_{10}[\text{H}^+]
\]

\subsection{Complejidad Computacional}

\textbf{Complejidades comunes:}
\begin{align*}
O(\log n) &: \text{Búsqueda binaria} \\
O(n \log n) &: \text{Merge sort, Quick sort (promedio)} \\
O(\log \log n) &: \text{Interpolation search (promedio)} \\
O(n^{\log n}) &: \text{Complejidad super-polinomial}
\end{align*}

\subsection{Número de Dígitos}

El número de dígitos de $n$ en base $b$ es:
\[
\lfloor \log_b(n) \rfloor + 1
\]

\textbf{Ejemplos:}
\begin{align*}
\text{Dígitos decimales de } n &= \lfloor \log_{10}(n) \rfloor + 1 \\
\text{Bits de } n &= \lfloor \log_2(n) \rfloor + 1
\end{align*}

\section{Trucos y Técnicas}

\subsection{Linearización de Productos}

Para calcular productos grandes:
\[
\prod_{i=1}^{n} a_i = \exp\left(\sum_{i=1}^{n} \ln(a_i)\right)
\]

Útil para evitar overflow/underflow.

\subsection{Comparar Potencias}

Para comparar $a^b$ vs $c^d$:
\[
a^b > c^d \iff b\ln(a) > d\ln(c)
\]

\subsection{Resolución de Recurrencias}

Para recurrencias del tipo $T(n) = aT(n/b) + f(n)$:

El término $\log_b(a)$ aparece en el Teorema Maestro.

\subsection{Cambio de Base Útil}

\[
\log_2(n) = \frac{\ln(n)}{\ln(2)} \approx 1.4427 \ln(n)
\]

\[
\log_{10}(n) = \frac{\ln(n)}{\ln(10)} \approx 0.4343 \ln(n)
\]

\[
\log_2(n) = \frac{\log_{10}(n)}{\log_{10}(2)} \approx 3.3219 \log_{10}(n)
\]

\section{Valores Numéricos Importantes}

\subsection{Constantes}

\begin{align*}
e &\approx 2.71828182845904523536 \\
\ln(2) &\approx 0.69314718055994530942 \\
\ln(10) &\approx 2.30258509299404568402 \\
\log_{10}(e) &\approx 0.43429448190325182765 \\
\log_2(e) &\approx 1.44269504088896340736 \\
\log_2(10) &\approx 3.32192809488736234787
\end{align*}

\subsection{Logaritmos de Números Pequeños}

\begin{align*}
\ln(2) &\approx 0.693 \\
\ln(3) &\approx 1.099 \\
\ln(5) &\approx 1.609 \\
\ln(7) &\approx 1.946 \\
\ln(10) &\approx 2.303
\end{align*}

\subsection{Logaritmos Base 2}

\begin{align*}
\log_2(3) &\approx 1.585 \\
\log_2(5) &\approx 2.322 \\
\log_2(10) &\approx 3.322 \\
\log_2(e) &\approx 1.443
\end{align*}

\section{Fórmulas Misceláneas}

\subsection{Identidades con Sumas y Productos}

\[
\log\left(\prod_{i=1}^{n} a_i\right) = \sum_{i=1}^{n} \log(a_i)
\]

\[
\prod_{i=1}^{n} a_i^{b_i} = \exp\left(\sum_{i=1}^{n} b_i \ln(a_i)\right)
\]

\[
\ln\left(\frac{1}{n!}\right) = -\ln(n!) = -\sum_{k=1}^{n} \ln(k)
\]

\subsection{Media Geométrica}

La media geométrica de $a_1, a_2, \ldots, a_n$ es:
\[
\sqrt[n]{a_1 a_2 \cdots a_n} = \exp\left(\frac{1}{n}\sum_{i=1}^{n} \ln(a_i)\right)
\]

\subsection{Relación con Sumatorias}

\[
\sum_{k=1}^{n} \ln(k) = \ln(n!)
\]

\[
\sum_{k=1}^{n} \frac{1}{k} = H_n \approx \ln(n) + \gamma
\]

donde $\gamma \approx 0.5772$ es la constante de Euler-Mascheroni.

\section{Tabla de Referencia Rápida}

\begin{center}
\begin{tabular}{|l|c|}
\hline
\textbf{Propiedad} & \textbf{Fórmula} \\
\hline
Producto & $\log(xy) = \log(x) + \log(y)$ \\
\hline
Cociente & $\log(x/y) = \log(x) - \log(y)$ \\
\hline
Potencia & $\log(x^n) = n\log(x)$ \\
\hline
Cambio de base & $\log_b(x) = \frac{\ln(x)}{\ln(b)}$ \\
\hline
Recíproco & $\log_b(a) = \frac{1}{\log_a(b)}$ \\
\hline
Derivada & $\frac{d}{dx}[\ln(x)] = \frac{1}{x}$ \\
\hline
Serie & $\ln(1+x) = \sum_{n=1}^{\infty} \frac{(-1)^{n+1}x^n}{n}$ \\
\hline
Límite & $\lim_{x \to 0} \frac{\ln(1+x)}{x} = 1$ \\
\hline
Desigualdad & $\ln(x) \leq x - 1$ \\
\hline
\end{tabular}
\end{center}

\section{Gráficas y Comportamiento}

\subsection{Características de $y = \log_b(x)$}

\begin{itemize}
    \item Dominio: $(0, \infty)$
    \item Rango: $(-\infty, \infty)$
    \item Intercepto con eje $x$: $(1, 0)$
    \item Asíntota vertical: $x = 0$
    \item Creciente si $b > 1$, decreciente si $0 < b < 1$
    \item Cóncava hacia abajo
\end{itemize}

\subsection{Puntos Importantes}

\begin{align*}
\log_b(1) &= 0 \\
\log_b(b) &= 1 \\
\log_b(b^2) &= 2 \\
\log_b(1/b) &= -1 \\
\log_b(\sqrt{b}) &= 1/2
\end{align*}

\section{Conclusión}

Este cheatsheet cubre las propiedades más importantes de los logaritmos utilizadas en matemáticas, ciencias e ingeniería. Los logaritmos son fundamentales para:

\begin{itemize}
    \item Simplificar cálculos con números muy grandes o muy pequeños
    \item Resolver ecuaciones exponenciales
    \item Analizar complejidades algorítmicas
    \item Representar escalas de magnitud (decibeles, Richter, pH)
    \item Calcular entropía y teoría de la información
    \item Trabajar con productos y tasas de crecimiento
\end{itemize}

\vspace{2em}
\begin{center}
\rule{0.8\textwidth}{0.4pt}

\textbf{¡Domina los logaritmos y dominarás el análisis matemático!}

\rule{0.8\textwidth}{0.4pt}
\end{center}

\end{document}
