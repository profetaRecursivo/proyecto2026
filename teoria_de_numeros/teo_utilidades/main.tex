\documentclass[12pt]{article}

% Paquetes necesarios
\usepackage{amsthm}  % Para los entornos theorem, definition, etc.
\usepackage{amsmath}  % Para matemáticas avanzadas
\usepackage{amssymb}  % Para símbolos matemáticos
\usepackage{geometry}  % Para personalizar márgenes
\usepackage{graphicx}  % Para incluir imágenes
\usepackage{hyperref}  % Para crear hipervínculos
\usepackage{fancyhdr}  % Para encabezados personalizados
\usepackage{enumitem} % Paquete para personalizar listas

% Configuración de márgenes
\geometry{a4paper, margin=1in}

% Aumentamos la altura del encabezado para evitar el error
\setlength{\headheight}{15pt}

% Definición de los entornos theorem y definition
\newtheorem{theorem}{Teorema}[section]
\newtheorem{definition}[theorem]{Definición}
\newtheorem{example}[theorem]{Ejemplo}

% Encabezado
\pagestyle{fancy}
\fancyhead[L]{Teoría de Números}
\fancyhead[R]{\thepage}

\title{Teoría de Números}
\author{Jaime Sebastian Chavarria Fuertes}
\date{\today}

\newlist{propiedades}{itemize}{1} % Basado en itemize
\setlist[propiedades]{label=--, left=0pt, itemsep=0.5em} % Configuración personalizada

\begin{document}

\maketitle

\section{Aritmética Modular}

\subsection{Teoría de Congruencias}

Una congruencia es una relación de equivalencia entre enteros que se basa en sus restos al dividirse por un número dado.

\begin{definition}[Congruencia]
Decimos que dos números enteros \( a \) y \( b \) son congruentes módulo \( n \), y escribimos \( a \equiv b \pmod{n} \), que significa que: \[
a \% n == b \% n
\]
Tambien podemos decir que si \( a \equiv b \pmod{n} \) entonces se puede decir que:
\[ n | (a-b)\]
Es decir que \( n \) divide a: \( ( a - b ) \)
\end{definition}

\begin{example}
\( 17 \equiv 2 \pmod{5} \), ya que \( 17 - 2 = 15 \) es divisible por 5.
\end{example}
\textbf{Propiedades: }
\begin{propiedades}
    \item Si \( a \equiv b \pmod{n} \), entonces \( a + x \equiv b + x \pmod{n} \).
    \item Si \( a \equiv b \pmod{n} \), entonces \( a \cdot x \equiv b \cdot x \pmod{n} \).
    \item Si \( a \equiv b \pmod{n} \) y \( x \equiv y \pmod{n} \), entonces \( a + x \equiv b + y \pmod{n} \).
    \item \(a^{n} \equiv b^{n} \pmod{m}\)  \( \forall n \in \mathbb{N} \)
    \item Dispuesto a completar y/o agregar cosas
\end{propiedades}

\section{Numeros Primos}
\begin{definition}[Numero Primo]
    Un numero es primos si y solo si tiene unicamente dos divisores
\end{definition}




\section{Funciones Aritméticas}

Las funciones aritméticas son aquellas que se definen en los números enteros y tienen aplicaciones importantes en teoría de números.

\begin{definition}[Función \(\phi\) de Euler]
La función \(\phi(n)\) cuenta el número de enteros positivos menores o iguales que \( n \) que son coprimos con \( n \).
\end{definition}

\begin{theorem}[Propiedad de la función \(\phi\)]
Si \( p_1, p_2, \dots, p_k \) son los factores primos distintos de \( n \), entonces:
\[
\phi(n) = n \left(1 - \frac{1}{p_1}\right) \left(1 - \frac{1}{p_2}\right) \dots \left(1 - \frac{1}{p_k}\right)
\]
\end{theorem}

\begin{example}
Para \( n = 12 \), los factores primos son 2 y 3, por lo que:
\[
\phi(12) = 12 \left(1 - \frac{1}{2}\right)\left(1 - \frac{1}{3}\right) = 12 \times \frac{1}{2} \times \frac{2}{3} = 4
\]
\end{example}

\section{Conclusiones}

La teoría de números es una de las ramas más antiguas de las matemáticas y sigue siendo un área activa de investigación. Desde su aplicación en criptografía hasta su influencia en otras ramas matemáticas, los resultados de esta teoría continúan jugando un papel crucial.

\end{document}
